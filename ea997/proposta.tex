\documentclass{article}

\usepackage[utf8]{inputenc}
\usepackage[T1]{fontenc}
\usepackage[portuguese]{babel}
\usepackage{amsmath}
\usepackage{amssymb}
\usepackage{bbm}
\usepackage{float}
\usepackage{subcaption}
\usepackage{dirtytalk}
\usepackage{url}
\usepackage[a4paper, total={6in, 8in}]{geometry}
\geometry{margin=1in}

\title{Projeto EA997 - Electrical Muscle Stimulation}
\author{
    Rodrigo Seiji Piubeli Hirao - 186837
}
\date{\today}

\begin{document}
    \maketitle
    \abstract{
        O projeto visa criar um aparelho que irá aplicar uma certa corrente
        no braço afim de controlar seu movimento.
    }
    \section{Introdução}
        EMS (Eletrical Muscle Stimulation) ou 
        NMES (Neuromuscular Eletrical Stimulation) consiste em
        um sistema que produz uma corrente em uma fibra muscular contraindo-a.
        
        Atualmente sua aplicação está ligada a:
        \begin{itemize}
             \item treino de musculatura
             \item rehabilitação
             \item testar o funcionamento fo funcionamento neural/muscular
             \item recuperação após exercícios físicos intensos
        \end{itemize}

        O projeto não irá focar em nenhuma dessas aplicações a princípio
    \section{Estado da Arte}
        Atualmente existem muitos EMS's criados até para uso não comercial 
        e portátil. Foi testado o produto EW433 da NAiS, que limita a tensão
        para o máximo de 64V, tendo uma corrente eficaz de 4.5A e funcionando de 
        1 até 50Hz.

    \section{Metodologia}
        Será usado um microcontrolador com um amplificador como um 
        gerador de pulsos conectado a diversos eletrodos feito de uma 
        pequena placa metálica que serão encostados na pele do braço do
        indivíduo.

    \section{Motivação}
        A introdução para engenharia biomédica despertou um interesse
        no funcionamento elétrico das células, quero expandir esse 
        projeto depois para um entendimento cada vez maior do corpo humano 
        como um sistema elétrico, visando assim criar novas ferramentas
        para o melhor controle e tratamento de problemas relacionadas a isso.

        O EMS foi escolhido por sua utilização como um aparelho para 
        rehabilitação, o que pareceu muito interessante.

    \newpage
    \section{Referências}
        \begin{enumerate}
            \item European Journal of Applied Physiology.
                Electrical stimulation for neuromuscular testing and training: state-of-the art and unresolved issues 
                \url{https://link.springer.com/article/10.1007%2Fs00421-011-2133-7}
            \item J Neuroeng Rehabil.
                A Neuromuscular Electrical Stimulation (NMES) and robot hybrid system for multi-joint coordinated upper limb rehabilitation after stroke
                \url{https://www.ncbi.nlm.nih.gov/pmc/articles/PMC5406922/}
            \item Andrew Bulman, Andrew Mastele, Patrick Deeney, Brian Narby.
                Neuromuscular Electrical Stimulator Design
                \url{www.engineering.pitt.edu/Sub-Sites/Labs/Sharma-Lab/_Library/Spring2014_3/}
        \end{enumerate}

\end{document}