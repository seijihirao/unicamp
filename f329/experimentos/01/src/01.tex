\documentclass{article}

\usepackage[utf8]{inputenc}
\usepackage[T1]{fontenc}
\usepackage[portuguese]{babel}
\usepackage{amsmath}
\usepackage{bbm}
\usepackage{circuitikz}

\title{Lei de Ohm e curva característica do diodo}
\author{
    Eduardo Parducci - 170272
    \and
    Lucas Koiti Geminiani Tamanaha - 182579
    \and
    Rodrigo Seiji Piubeli Hirao - 186837
    \and
    Tanus Vaz Szabo - 187308
}
\date{\today}

\begin{document}
    \maketitle
    \newpage
    \tableofcontents
    \newpage
    \listoffigures
    \newpage
    \section{Circuitos}
    \begin{figure} [h!] 
        \begin{circuitikz} \draw
        (0,0) to[battery=$V$]   (0,4)
        (0,4) to[R=$R_p$, o-o]    (4,4)
        (4,4) to[voltmeter] (4,0)
        (0,0) to[ammeter]   (4,0)
        (4,4) --            (8,4)
        (8,4) to[R=$R$, o-o]    (8,0)
        (4,0) --            (8,0)
        ;
        \end{circuitikz}
        \caption{Circuito 1}
        \label{fig:circuit1}
    \end{figure}
    \begin{figure}[h!] 
        \begin{circuitikz} \draw
        (0,0) to[battery=$V$]   (0,4)
        (0,4) to[R=$R_p$, o-o]    (4,4)
        (4,4) to[voltmeter] (4,0)
        (0,0) --            (4,0)
        (4,4) --            (8,4)
        (8,4) to[R=$R$, o-o]    (8,0)
        (4,0) to[ammeter]   (8,0)
        ;
        \end{circuitikz}
        \caption{Circuito 2}
        \label{fig:circuit1}
    \end{figure}
\end{document}