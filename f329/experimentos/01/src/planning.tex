\documentclass{article}

\usepackage[utf8]{inputenc}
\usepackage[T1]{fontenc}
\usepackage[portuguese]{babel}
\usepackage{amsmath}
\usepackage{bbm}
\usepackage{circuitikz}

\title{Lei de Ohm e curva característica do diodo}
\author{
    Eduardo Parducci - 170272
    \and
    Lucas Koiti Geminiani Tamanaha - 182579
    \and
    Rodrigo Seiji Piubeli Hirao - 186837
    \and
    Tanus Vaz Szabo - 187308
}
\date{\today}

\begin{document}
    \maketitle
    \newpage
    \tableofcontents
    \newpage
    \listoffigures
    \newpage
    \section{Material Utilizado}
        \begin{itemize}
            \item 1 Resistor de $100\Omega$
            \item 1 Resistor de $10\Omega$
            \item 1 Resistor de $220\Omega$
            \item 2 multímetros
            \item 1 Protoboard
            \item 1 Diodo de silício
            \item 1 Fonte de tensão contínua
            \item Cabos de plug "banana"
        \end{itemize}
    \section{Procedimento}
        \subsection{Determinar resistências}
            Determinar o valor dos resisttores de 
            $10\Omega, 100\Omega e 220\Omega$, 
            com o uso do Multímetro na escala $\Omega$ 
            juntamente com sua incerteza e comparar os valores com o nominal.
            \newline
            Determinar os valores mínimo e máximo de tensão para a realização do experimento.
        \subsection{Curva do Resistor}
            Montar o Circuito \ref{fig:circuitL} 
            utilizando $R_p = 10\Omega$ e $R = 100\Omega$.
            \newline
            Realizar 24 medidas de tensão e corrente aumentando 
            a tensão gradativamente em 0.5V, sendo $V_{min}=0V$ e $V_{max}=12V$. 
            \newline
            Colocar os dados numa tabela com os seus valores e respectivos erros $V\pm\Delta V$
            e $i\pm\Delta i$
        \subsection{Curva do Diodo (Polarização Direta)}
            Montar o Circuito \ref{fig:circuitDiode} 
            utilizando $R_p = 220\Omega$.
            \newline
            Realizar 20 medidas de tensão e corrente aumentando 
            a tensão gradativamente em 0.5V, sendo $V_{min}=0V$ e $V_{max}=10V$. 
            \newline
            Colocar os dados numa tabela com os seus valores e respectivos erros $V\pm\Delta V$
            e $i\pm\Delta i$
        \subsection{Curva do Diodo (Polarização Reversa)}
            Montar o Circuito \ref{fig:circuitDiodeR} 
            utilizando $R_p = 10\Omega$.
            \newline
            Realizar 20 medidas de tensão e corrente aumentando 
            a tensão gradativamente em 0.5V, sendo $V_{min}=0V$ e $V_{max}=10V$. 
            \newline
            Colocar os dados numa tabela com os seus valores e respectivos erros 
            $V\pm\Delta V$ e $i\pm\Delta i$
    \section{Circuitos}
        \begin{figure} [h!] 
            \centering
            \begin{circuitikz} \draw
            (0,0) to[battery=$V$]   (0,4)
            (0,4) to[R=$R_p$, o-o]  (4,4)
            (4,4) to[voltmeter]     (4,0)
            (0,0) to[ammeter]       (4,0)
            (4,4) --                (8,4)
            (8,4) to[R=$R$, o-o]    (8,0)
            (4,0) --                (8,0)
            ;
            \end{circuitikz}
            \caption{Circuito para medição de resistências pequenas}
            \label{fig:circuitL}
        \end{figure}
        \begin{figure}[h!]  
            \centering
            \begin{circuitikz} \draw
            (0,0) to[battery=$V$]   (0,4)
            (0,4) to[R=$R_p$, o-o]  (4,4)
            (4,4) to[voltmeter]     (4,0)
            (0,0) --                (4,0)
            (4,4) --                (8,4)
            (8,4) to[R=$R$, o-o]    (8,0)
            (4,0) to[ammeter]       (8,0)
            ;
            \end{circuitikz}
            \caption{Circuito para medição de resistências grandes}
            \label{fig:circuitH}
        \end{figure}
        \begin{figure}[h!]  
            \centering
            \begin{circuitikz} \draw
            (0,0) to[battery=$V$]   (0,4)
            (0,4) to[R=$R_p$, o-o]  (4,4)
            (4,4) to[voltmeter]     (4,0)
            (0,0) to[ammeter]       (4,0)
            (4,4) --                (8,4)
            (8,4) to[Do]            (8,0)
            (4,0) --                (8,0)
            ;
            \end{circuitikz}
            \caption{Circuito de montagem do diodo na polarização direta}
            \label{fig:circuitDiode}
        \end{figure}
        \begin{figure}[h!]  
            \centering
            \begin{circuitikz} \draw
            (0,0) to[battery=$V$]   (0,4)
            (0,4) to[R=$R_p$, o-o]  (4,4)
            (4,4) to[voltmeter]     (4,0)
            (0,0) --                (4,0)
            (4,4) --                (8,4)
            (8,0) to[Do]            (8,4)
            (4,0) to[ammeter]       (8,0)
            ;
            \end{circuitikz}
            \caption{Circuito de montagem do diodo na polarização reversa}
            \label{fig:circuitDiodeR}
        \end{figure}
\end{document}