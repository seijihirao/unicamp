\documentclass{article}

\usepackage[utf8]{inputenc}
\usepackage[T1]{fontenc}
\usepackage[portuguese]{babel}
\usepackage{amsmath}
\usepackage{bbm}
\usepackage{circuitikz}

\title{Introdução ao osciloscópio}
\author{
    Eduardo Parducci - 170272
    \and
    Lucas Koiti Geminiani Tamanaha - 182579
    \and
    Rodrigo Seiji Piubeli Hirao - 186837
    \and
    Tanus Vaz Szabo - 187308
}
\date{\today}

\begin{document}
    \maketitle
\newpage
\tableofcontents
\newpage
    \section{Material Utilizado}
        \begin{itemize}
    \item [1] Bobina de Helmholtz 
        \begin{itemize}
             \item [n] 140 espiras
             \item [D] 22,3cm
             \item [d] 20,2cm
             \item [l] 1,13cm
             \item [L] 10,26cm
        \end{itemize}
    \item [1] Bússola
    \item [1] Ímã permanente cilíndrico 
        \begin{itemize}
             \item [D] 0,6cm
             \item [l] 2,53cm
             \item [m] 5,1616g
        \end{itemize}
    \item [1] Multímetro
    \item [1] Resistor de potência
    \item [1] Fonte de alimentação
    \item [1] Cronômetro
    \item [6] Fios de ligação
\end{itemize}
    \section{Procedimento}
        \subsection{Determinar resistências}
            Determinar o valor dos resisttores de 
            $10\Omega, 100\Omega e 220\Omega$, 
            com o uso do Multímetro na escala $\Omega$ 
            juntamente com sua incerteza e comparar os valores com o nominal.
            \newline
            Determinar os valores mínimo e máximo de tensão para a realização do experimento.
        \subsection{Curva do Resistor}
            Montar o Circuito \ref{fig:circuitL} 
            utilizando $R_p = 10\Omega$ e $R = 100\Omega$.
            \newline
            Realizar 24 medidas de tensão e corrente aumentando 
            a tensão gradativamente em 0.5V, sendo $V_{min}=0V$ e $V_{max}=12V$. 
            \newline
            Colocar os dados numa tabela com os seus valores e respectivos erros $V\pm\Delta V$
            e $i\pm\Delta i$
        \subsection{Curva do Diodo (Polarização Direta)}
            Montar o Circuito \ref{fig:circuitDiode} 
            utilizando $R_p = 220\Omega$.
            \newline
            Realizar 20 medidas de tensão e corrente aumentando 
            a tensão gradativamente em 0.5V, sendo $V_{min}=0V$ e $V_{max}=10V$. 
            \newline
            Colocar os dados numa tabela com os seus valores e respectivos erros $V\pm\Delta V$
            e $i\pm\Delta i$
        \subsection{Curva do Diodo (Polarização Reversa)}
            Montar o Circuito \ref{fig:circuitDiodeR} 
            utilizando $R_p = 10\Omega$.
            \newline
            Realizar 20 medidas de tensão e corrente aumentando 
            a tensão gradativamente em 0.5V, sendo $V_{min}=0V$ e $V_{max}=10V$. 
            \newline
            Colocar os dados numa tabela com os seus valores e respectivos erros 
            $V\pm\Delta V$ e $i\pm\Delta i$
    \section{Circuitos}
        \begin{figure} [H]
    \def\x{6}
    \def\y{6}

    \def\dx{3}
    \def\dy{3}

    \begin{circuitikz}

        \draw (0,0) to [battery=$12V$](0, \y) to (\x, \y)
        to [R=100$\Omega$, -] (\x-\dx,\y-\dy)
        to [R=100$\Omega$, -*] (\x,\y-2*\dy);

        \draw (\x,\y)
        to [R=68$\Omega$, -*] (\x+\dx, \y-\dy)
        to [R, l_=$R_{v}$, -*] (\x,\y-2*\dy)
        to (\x, 0) to (0,0);

        % Voltimetro
        \draw (\x-\dx, \y-\dy) to [voltmeter] (\x+\dx, \y-\dy);
    \end{circuitikz}

    \caption{Circuito com Ponte de Wheatstone}
    \label{fig:circuitW}
\end{figure}
\end{document}