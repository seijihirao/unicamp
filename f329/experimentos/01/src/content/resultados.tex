\section{Resultados}
    \subsection{Resistor ixU}
        \begin{figure} [H] 
            \centering
            \begin{tabular}{||c | c||}
                \hline
                U(V)    &   i(A)    \\
                \hline
                0.001   &   0.001   \\
                0.506   &   0.00489 \\
                1.014   &   0.01004 \\
                1.503   &   0.01497 \\
                1.944   &   0.01957 \\
                2.533   &   0.02549 \\
                3.004   &   0.03002 \\
                3.528   &   0.03551 \\
                3.951   &   0.03978 \\
                4.517   &   0.04541 \\
                4.929   &   0.04940 \\
                5.415   &   0.05450 \\
                5.916   &   0.05916 \\
                6.460   &   0.06490 \\
                6.98    &   0.07001 \\
                7.55    &   0.07580 \\
                7.99    &   0.08020 \\
                8.48    &   0.08520 \\
                9.06    &   0.09090 \\
                9.45    &   0.09490 \\
                10.10   &   0.10140 \\
                \hline
            \end{tabular}
            \caption{Tabela de dados da corrente adiquirida ao aumentar tensão em resistor}
            \label{fig:tableR}
        \end{figure}

        \begin{figure} [H] 
            \centering
            \begin{tikzpicture}
                \begin{axis}[
                    width=14cm,
                    xlabel={$U[V]$},
                    ylabel={$i[A]$},
                    xlabel style={below right},
                    ylabel style={above left},
                    ]
                    \addplot [color=cyan, mark=o, smooth, ultra thick]
                        plot [error bars/.cd, y dir = both, y explicit]
                        table[x =x, y =y]{data/resistor.dat};
                \end{axis}
            \end{tikzpicture}
            \caption{Gráfico da corrente adiquirida ao aumentar tensão em resistor}
            \label{fig:graphR}
        \end{figure}

    \subsection{Diodo ixU}
        \begin{figure} [H] 
            \centering
            \begin{tabular}{||c | c||}
                \hline
                U(V)    &   i(A)    \\
                \hline
                0       &   -0.5    \\
                0.27    &   0.02    \\
                0.47    &   0.12    \\
                0.55    &   0.16    \\
                0.59    &   1.13    \\
                0.65    &   6.73    \\
                0.67    &   8.47    \\
                0.70    &   17.18   \\
                0.71    &   25.24   \\
                0.72    &   26.24   \\
                0.73    &   43.15   \\
                0.74    &   43.15   \\
                \hline
            \end{tabular}
            \caption{Gráfico da corrente adiquirida ao aumentar tensão em diodo}
            \label{fig:tableD}
        \end{figure}

        \begin{figure} [H] 
            \centering
            \begin{tikzpicture}
                \begin{axis}[
                    width=14cm,
                    xmode=log,
                    xlabel={$U[V]$},
                    ylabel={$i[mA]$},
                    xlabel style={below right},
                    ylabel style={above left},
                    ]
                    \addplot [color=cyan, mark=o, smooth, ultra thick]
                        plot [error bars/.cd, y dir = both, y explicit]
                        table[x =x, y =y, x error=ex, y error=ey]{data/diodo.dat};
                \end{axis}
            \end{tikzpicture}
            \caption{Gráfico da corrente adiquirida ao aumentar tensão em diodo}
            \label{fig:graphD}
        \end{figure}


    \subsection{Diodo RxU}
        \begin{figure} [H] 
            \centering
            \begin{tabular}{||c | c||}
                \hline
                U(V)    &$R(\Omega)$\\
                \hline
                -0.5    &   0       \\     
                0.27    &   13.50   \\ 
                0.47    &   3.910   \\ 
                0.59    &   0.522   \\ 
                0.65    &   0.096   \\ 
                0.67    &   0.079   \\ 
                0.70    &   0.040   \\ 
                0.71    &   0.028   \\ 
                0.72    &   0.027   \\ 
                0.73    &   0.016   \\
                0.74    &   0.017   \\
                \hline
            \end{tabular}
            \caption{Gráfico da corrente adiquirida ao aumentar tensão em diodo}
            \label{fig:tableD2}
        \end{figure}

        \begin{figure} [H] 
            \centering
            \begin{tikzpicture}
                \begin{axis}[
                    width=14cm,
                    xmode=log,
                    xlabel={$U[V]$},
                    ylabel={$R[\Omega]$},
                    xlabel style={below right},
                    ylabel style={above left},
                    ]
                    \addplot [color=cyan, mark=o, smooth, ultra thick]
                        plot [error bars/.cd, y dir = both, y explicit]
                        table[x =x, y =y, x error=ex, y error=ey]{data/diodo2.dat};
                \end{axis}
            \end{tikzpicture}
            \caption{Gráfico da corrente adiquirida ao aumentar tensão em diodo}
            \label{fig:graphD2}
        \end{figure}
        