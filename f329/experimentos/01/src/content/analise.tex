\section{Análise de Dados}
    Com os dados obtidos na tabela da figura \ref{fig:tableR} podemos construir o gráfico da figura \ref{fig:graphR}.

    Através da relação entre a tensão e a corrente, podemos linearizar o gráfico acima, respeitando a relação $U = R\times I$
    utilizando o método dos mínimos quadrados juntamente com propagação de erros para calcular sua incerteza, a fim de obtermos
    uma relação no formato $(R_{exp}\pm \Delta R_{exp})\Omega$. Obtendo o gráfico da figura \ref{fig:graphMMQ}, com o Método dos Mínimos Quadrados:
    \newline
    $a = \overline{y} - b\overline{x}$ 
    \newline
    $b = \frac{\sum_{i=1}^{n}x_i(y_i-\overline{y})}{\sum_{i=1}^{n}x_i(x_i-\overline{x})}$ 
    \newline

    Finalmente, temos $R_{exp} = (99,9\pm 0,1)\Omega$

    O diodo apresentou comportamento diferenciado, nota-se abaixo uma curva exponencial, impedindo a passagem de corrente quando
    aplicado a uma tensão negativa, e possibilitando a passagem de corrente para tensões acima de $0,6V$, visível no gráfico da figura \ref{fig:graphD}

    Novamente através da lei de Ôhm podemos construir o gráfico da resistência do diodo para cada ponto de medição, com sua devida
    incerteza calculada através da propagação de erros da seguinte forma:
    $\sigma_{R_{diodo}}=\sqrt{(\frac{\partial R}{\partial V})^2 \sigma_{V}^2 + (\frac{\partial R}{\partial i})^2 \sigma_{i}^2} $, 
    como na figura \ref{fig:graphD2}.
