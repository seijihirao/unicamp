\section{Discussão}
    Os resultados obtidos comprovam o fato do resistor 
    utilizado ser ôhmico, sendo isso visível pela figura 
    \ref{fig:graphR} que mostrou uma função linear com o aumento da tensão.
    \newline
    A figura \ref{fig:graphD} comprovou que o diodo apresenta um comportamento
    exponencial com o aumento da tensão, o que significa que o diodo não é ôhmico.
    Além disso ele apresenta corrente nula ao passar uma tensão menor que 0V, ou seja,
    ao usá-lo na polarização reversa. Demonstrando características de um componente retificador.

    

