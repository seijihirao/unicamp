\section{Geral}

    \subsection{Efeitos da resistência interna}
        
        Sabendo que existe uma resistência interna no gerador de ondas
        podemos calcular $\tau$ da seguinte forma:

        $$\tau = (R+R_{int})*C$$

        Dessa forma observamos que, se $R >> R_{int}$, a interferência da 
        resistência interna pode ser desprezada, como acontece nas partes
        2 e 3 (onde $R = 10 k\Omega$). Porém na parte 1 temos que 
        $R_{int} \approx 0,2R$, ocasionando uma interferência significativa
        que pode ser observada no cálculo de $\tau_{teorico}$

    \subsection{Fontes de Erro}

        Bem como as fontes anteriormente citadas (resitências internas
        e capacitâncias parasitas) podemos ter uma variação das distâncias d
        do capacitor montado no laboratório, devido à variação de pressão
        ao fixar o papel entre as placas de alumínio, fator que não foi levado
        em consideração nos cálculos desse experimento.