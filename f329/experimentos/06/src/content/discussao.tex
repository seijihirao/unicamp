\section{Discussão}
    Os resultados obtidos apresentam notáveis diferenças daqueles esperados, 
    dessa forma, pode-se apontar alguns motivos experimentais para isso. 
    O momento magnético do ímã $(\mu = 0,24 \pm 0,03) A m^2$ apresentou-se perto, 
    porém pouco conclusivo. O campo magnético da terra esperado é de 2x10-5T, 
    no entanto, foi determinado como $(1,44 \pm 0,07) \times 10^{-5}T$, mesmo ambos os valores 
    tendo ordem de grandeza iguais, não pode-se confirmar que o resultado
    foi convincente, pois o erro não abrange o teórico, e essa diferença para 
    campo magnético representa uma variação geográfica muito grande. 
    Esse resultado pode ser causado pelas dificuldades ao realizar o 
    experimento, tais como alinhar as bobinas ao campo da Terra, observar 
    as variações de ângulo e assim o seu erro, o qual influencia 
    significativamente no resultado, tanto como a frequência das oscilações 
    que foram medidas manualmente com um cronômetro.