\section{Conclusão}
    Ao comparar o valor obtido do campo magnético da Terra $B_{Terra}$ 
    obtidos pelo gráfico [$f^2$ x I] com o esperado pela teoria, 
    vimos que os valores se distinguem, tal fato pode ter sido 
    dado pelo fato da interferência de um campo magnético gerado pela 
    corrente elétrica que passavam nos equipamentos utilizados no 
    experimento (multímetro, fonte de tensão, os cabos que conectam 
    os equipamentos à rede elétrica)  no da Terra e da espira. 
    Tal campo era visível quando, ao tentar encontrar o pólo norte 
    geográfico para o alinhamento das espiras, ocorrem interferências 
    na agulha da bússola, mostrando que havia um campo magnético além 
    do terrestre. Dessa forma, pode-se concluir que o experimento não 
    foi conclusivo o bastante, pois essa pequena diferença entre o 
    campo magnético da Terra esperado e o experimental, representa uma 
    variação geográfica no globo muito grande, na escala de diferença de 
    país em que ocorreu.