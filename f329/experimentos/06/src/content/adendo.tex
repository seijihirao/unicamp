\section{Adendo}
    \paragraph{Otra forma} de encontrar um valor para o campo magnético local, 
    usamos uma bússola e as espiras, encontrando assim o campo

    $$B_{Terra} = (1,8820,004)\times 10^{-5}T$$
    
    Este valor foi encontrado colocando deixando as espiras perpendiculares
     ao campo magnético terrestre, a bússola no centro das espiras, 
     desta forma a bússola apontaria para o campo magnético resultante do 
     campo terrestre e das espiras. Portanto, variamos a corrente do 
     circuito até obter uma mudança na bússola de 45º, dado que 
     inicialmente não havia corrente nas espiras e, portanto, só havia 
     atuação do campo terrestre. Sendo a bússola variada em 45º, temos 
     que o campo criado pelas espiras é igual ao campo terrestre. Portanto:
    
    $$B_{Terra} = B_{Helmholtz} = \frac{8 \mu_0 I N}{5^{\frac{3}{2}} R}$$
    
    Mas, esse valor encontrado pode ter uma incerteza maior do 
    que a calculada, devido ao atrito da agulha da bússola com sua 
    armação, ocorrendo uma incerteza no valor do ângulo na bússola, 
    que acarreta uma incerteza à igualdade entre o campo da espira e terrestre.
