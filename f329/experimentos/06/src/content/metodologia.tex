\section{Metodologia}
    \subsection{Material Utilizado}
        \begin{itemize}
    \item [1] Bobina de Helmholtz 
        \begin{itemize}
             \item [n] 140 espiras
             \item [D] 22,3cm
             \item [d] 20,2cm
             \item [l] 1,13cm
             \item [L] 10,26cm
        \end{itemize}
    \item [1] Bússola
    \item [1] Ímã permanente cilíndrico 
        \begin{itemize}
             \item [D] 0,6cm
             \item [l] 2,53cm
             \item [m] 5,1616g
        \end{itemize}
    \item [1] Multímetro
    \item [1] Resistor de potência
    \item [1] Fonte de alimentação
    \item [1] Cronômetro
    \item [6] Fios de ligação
\end{itemize}
    \subsection{Procedimento}
        \subsection{Preparo do ambiente}
            Primeiro deve-se alinhar a Bobina ao campo magnético da terra,
            com ajuda da bússola.
            E montar o circuito:
            \begin{figure} [H] 
    \centering
    \begin{circuitikz} \draw
    (0,0) to[battery=$V$]   (0,4)
    (0,4) to[R=$R_p$]  (4,4)
    (4,4) to[inductor]     (4,0)
    (0,0) to[ammeter]       (4,0)
    ;
    \end{circuitikz}
    \caption{Circuito ligado à bobina}
    \label{fig:circuit}
\end{figure}
        \subsection{Medidas}
            Primeiro foram medidas as dimenções das bobinas e do imã.
            
            Deixou-se o imã oscilando dentro do campo linear criado
            pela bobina, assim medindo seu período de oscilação.

            Foram feitas as medidas de 50 oscilações por vez em 10
            correntes diferentes (variando de 20 a 200mA)
        \subsection{Cálculo}
            Foi usado a equação:

            $$f^2 = \frac{\mu}{4 \pi^2 m_I} 
            [(\frac{8\mu_0 N}{5^{\frac{3}{2}}}) I \pm B_{Terra}]$$
            Liearizado para o cálculo da componente 
            horizontal do campo magnético terrestre 
            e o momento de dipolo magnético do ímã.