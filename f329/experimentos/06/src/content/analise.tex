\section{Análise de Dados}
    Após a medida dos dados foi utilizado a equação:

    $$m_I = m (\frac{r^2}{4} + \frac{L^2}{12})$$

    Propagando:

    $$\sigma_{m_I} = \sqrt{
        (\frac{r^2}{4} + \frac{L^2}{12})^2 \sigma_m^2 + 
        (m (\frac{r}{2} + \frac{L^2}{12}))^2 \sigma_r^2 + 
        (m (\frac{r^2}{4} + \frac{L}{6}))^2 \sigma_L^2 + 
    }$$


    Para descobrir o momento de inércia do ímã,
    e a equação:

    $$B_{Helmholtz} = 
    \frac{8\mu_0 I N}{5^{\frac{3}{2}}}$$

    adaptada para:

    $$f^2 = \frac{\mu}{4 \pi^2 m_I} 
    [(\frac{8\mu_0 N}{5^{\frac{3}{2}} R}) I \pm B_{Terra}]$$
    
    Para descobrirmos que o coeficiente angular é:

    $$a = \frac{2 \mu \mu_0 N}{5^{\frac{3}{2}}\pi^2 m_I} $$
    $$\mu = \frac{5^{\frac{3}{2}} R a \pi^2 m_I}{2 \mu_0 N} $$
    
    Propagando:

    $$\sigma_\mu = \sqrt{
        (\frac{5^{\frac{3}{2}} R \pi^2 m_I}{2 \mu_o N})^2 \sigma_a^2 + 
        (\frac{5^{\frac{3}{2}} R \pi^2 a}{2 \mu_o N})^2 \sigma_{m_I}^2 + 
        (\frac{5^{\frac{3}{2}}\pi^2 a m_I}{2 \mu_o N})^2 \sigma_{R}^2
    }$$

    E o coeficiente linear é:

    $$b = \pm \frac{\mu B_{Terra}}{4 \pi^2 m_I}$$
    $$B_{Terra} = \frac{|b| 4 \pi^2 m_I}{\mu}$$

    Propagando:

    $$\sigma_{B_{Terra}} = \sqrt{
        (\frac{|b| 4 \pi^2 m_I}{b \mu})^2 \sigma_b^2 +
        (\frac{|b| 4 \pi^2}{\mu})^2 \sigma_{m_I}^2 +
        (-\frac{|b| 4 \pi^2 m_I}{\mu^2})^2 \sigma_\mu^2
    }$$

    Assim conseguindo:

    $$m_I = (2,92 \pm 0,01) \times 10^{-7} kg m^2$$
    $$\mu = (2,43 \pm 0,01) \times 10^{-1} A m^2$$
    $$B_{Terra} = (1,44 \pm 0,07) \times 10^{-5} T$$