\documentclass{article}

\usepackage[utf8]{inputenc}
\usepackage[T1]{fontenc}
\usepackage[portuguese]{babel}
\usepackage{amsmath}
\usepackage{bbm}
\usepackage{circuitikz}
\usepackage{float}

\title{%
Regressão Linear \\
\large Projeto - MC613
}
\author{
    Rodrigo Seiji Piubeli Hirao - 186837
    \and
    Luiz Eduardo Araujo Zucchi - 183006
}
\date{\today}

\begin{document}
    \maketitle
\newpage
\tableofcontents
\newpage
\listoffigures
\newpage
    \section{Materiais}
        \begin{itemize}
    \item [1] Bobina de Helmholtz 
        \begin{itemize}
             \item [n] 140 espiras
             \item [D] 22,3cm
             \item [d] 20,2cm
             \item [l] 1,13cm
             \item [L] 10,26cm
        \end{itemize}
    \item [1] Bússola
    \item [1] Ímã permanente cilíndrico 
        \begin{itemize}
             \item [D] 0,6cm
             \item [l] 2,53cm
             \item [m] 5,1616g
        \end{itemize}
    \item [1] Multímetro
    \item [1] Resistor de potência
    \item [1] Fonte de alimentação
    \item [1] Cronômetro
    \item [6] Fios de ligação
\end{itemize}
    \section{Procedimento}
        \subsection{Preparo do ambiente}
            Primeiro deve-se alinhar a Bobina ao campo magnético da terra,
            com ajuda da bússola.
            E montar o circuito:
            \begin{figure} [H] 
    \centering
    \begin{circuitikz} \draw
    (0,0) to[battery=$V$]   (0,4)
    (0,4) to[R=$R_p$]  (4,4)
    (4,4) to[inductor]     (4,0)
    (0,0) to[ammeter]       (4,0)
    ;
    \end{circuitikz}
    \caption{Circuito ligado à bobina}
    \label{fig:circuit}
\end{figure}
        \subsection{Medidas}
            Primeiro serão medidas as dimenções das bobinas e do imã.
            
            Deixa-se o imã oscilando dentro do campo linear criado
            pela bobina, assim medindo seu período de oscilação.

            Serão feitas as medidas de 50 oscilações por vez em 10
            correntes diferentes (variando de 20 a 200mA)
\end{document}