\documentclass{article}

\usepackage[utf8]{inputenc}
\usepackage[T1]{fontenc}
\usepackage[portuguese]{babel}
\usepackage{amsmath}
\usepackage{amssymb}
\usepackage{bbm}
\usepackage{circuitikz}
\usepackage{pgfplots}
\usepackage{float}

\newcommand\tab[1][0.6cm]{\hspace*{#1}}

\title{Introdução ao osciloscópio}
\author{
    Eduardo Parducci - 170272
    \and
    Lucas Koiti Geminiani Tamanaha - 182579
    \and
    Rodrigo Seiji Piubeli Hirao - 186837
    \and
    Tanus Vaz Szabo - 187308
}
\date{\today}

\begin{document}
    \maketitle
\newpage
\tableofcontents
\newpage
    \section{Material Utilizado}
        \begin{itemize}
    \item [1] Bobina de Helmholtz 
        \begin{itemize}
             \item [n] 140 espiras
             \item [D] 22,3cm
             \item [d] 20,2cm
             \item [l] 1,13cm
             \item [L] 10,26cm
        \end{itemize}
    \item [1] Bússola
    \item [1] Ímã permanente cilíndrico 
        \begin{itemize}
             \item [D] 0,6cm
             \item [l] 2,53cm
             \item [m] 5,1616g
        \end{itemize}
    \item [1] Multímetro
    \item [1] Resistor de potência
    \item [1] Fonte de alimentação
    \item [1] Cronômetro
    \item [6] Fios de ligação
\end{itemize}
    \section{Procedimento}
        \subsection{Determinar resistência $R_{x}$ utilizando a ponte de Wheatstone}
            \tab Com o uso do multímetro na escala $\Omega$, checar o valor dos resistores nominais de
            $68\Omega$ e $100\Omega$ anotando os valores e respectivos erros comparando-os com o nominal.
            \par
            Determinar o valor da tensão para a montagem do circuito 1 de forma que $R_{v}$ varie entre $10\Omega$ e $60\Omega$
            fazendo com que potência dissipada em $R_{p}=100\Omega$ não ultrapasse $1,5W$.
            \par
            Montar o circuito 1 (Ponte de Wheatstone) e realizar 20 medições da tensão (multímetro na escala $V\backsimeq$ )
            variando $R_{v}$ entre $10\Omega$ e $60\Omega$. Colocar os valores e seus respectivos erros em uma tabela
            $R_{v}\pm\Delta R_{v}$ e $V\pm\Delta V$.\newline
            \textbf{Obs:}Diminuir a variação $\Delta R_{v}=0,1\Omega$ quando os valores da tensão estiverem próximos de zero a fim de obter um
            gráfico mais consistente.
        \subsection{Determinar os coeficientes de fabricação do termistor}
            \tab Anotar o número do termistor utilizado e montar o circuito 2 mantendo $R_{p}=100\Omega$.
            \par
            Colocar água e o aquecedor no Béquer de forma que a resistência do Aquecedor fique totalmente submersa na água.
            \par
            Ligar o Aquecedor e medir a temperatura da água, até atingir $T_{max}=333K$, desligar o aquecedor e realizar 30 medidas
            da tensão até atingir $T_{min}=303K$ obtendo uma leitura a cada $\Delta T = 1K$ colocando os valores numa tabela $T\pm\Delta T$
            e $V\pm\Delta V$.
    ``A equipe declara que este relatório que está sendo entregue foi escrito por ela e
que os resultados apresentados foram medidos por ela durante as aulas de F 329 no
1oS/2017. Declara ainda que o relatório contém um texto original que não foi submetido
anteriormente em nenhuma disciplina dentro ou fora da Unicamp.``
\end{document}