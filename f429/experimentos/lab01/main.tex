\documentclass[a4paper]{article}

%% Language and font encodings
\usepackage[utf8]{inputenc}
\usepackage[T1]{fontenc}
\usepackage[portuguese]{babel}

%% Sets page size and margins
\usepackage[
	a4paper,
    top=3cm,
    bottom=2cm,
    left=2cm,
    right=2cm,
    marginparwidth=1.75cm
]{geometry}

%%tabelas sofisticadas
\usepackage{booktabs}
\usepackage[table,xcdraw]{xcolor}

%% Pacotes úteis
\usepackage{amsmath}
\usepackage{graphicx}
\usepackage{subfigure} %pacote para  subfiguras
\usepackage[colorinlistoftodos]{todonotes}
\usepackage[colorlinks=true]{hyperref}
\usepackage{amsmath}
\usepackage{amssymb}
\usepackage{bbm}
\usepackage{pgfplots}
\usepackage{pgfplotstable}
\usepackage{float}
\usepackage{gensymb}

%% Desenhando circuitos
\usepackage{circuitikz}


\title{Resposta espectral de circuitos RC, RL, RLC}
\author{
    Gustavo de Noraes Furtado (150653)
    \and 
	Guilherme Hasse Urel (157946)
    \and
	João Pedro de Amorim (176131)
	\and 
    Rodrigo Seiji Piubeli Hirao (186837)
    }

\begin{document}
\maketitle

\begin{abstract}
Através do seguinte experimento evolvendo circuitos RC E RLC, foram observados filtros de circuitos elétricos tipo passa-baixa e tipo passa-banda, respectivamente. E com a análise dos dados, obtidos com o auxílio do osciloscópio, pode-se gerar os Diagramas de Bode, relacionando a transmitância e a fase do circuito em função da frequência.
\end{abstract}

\section{Introdução}
	Muitos dispositivos atuais fazem o uso de circuitos elétricos formados por elementos simples como resistores, indutores e capacitores. Tais elementos possibilitam a determinação de grandezas físicas básicas como a carga elétrica e suas derivadas: corrente elétrica e derivada da corrente elétrica. Devido às características dos elementos constituintes e seu arranjo, tais circuitos apresentam comportamentos de grande interesse, devido às inúmeras aplicações tecnológicas possíveis. 
	Ao analisarmos circuitos RLC (formados por resistores, indutores e capacitores) excitados por uma fonte de corrente alternada, podemos observar a grande dependência que o sinal de saída tem com relação a frequência de excitação. Por essa característica específica tais circuitos são empregados comumente como filtros de frequência, barrando ou não determinados sinais de acordo com a frequência dos mesmos. Visando explorar o funcionamento desses filtros, suas peculiaridades e também o aprofundamento nos conceitos que regem tais fenômenos, alguns problemas foram propostos para solução através do emprego de tais circuitos. A familiarização com os circuitos de corrente alternada de uma forma geral e alguns conceitos empregados para a sua análise também se deram através da solução desses casos. A análise de resposta em frequência apresenta-se então como uma poderosa ferramenta nesse contexto, para a caracterização dos parâmetros de interesse desses  sistemas. 

\section{Materiais e métodos}

Neste experimento, para a montagem dos filtros do tipo passa-baixa (a partir de um circuito RC) e passa-banda (a partir de um circuito RLC), utilizamos:

\subsection{Componentes do circuito passa-baixa e equipamentos:}

\begin{itemize}
\item \textbf{Resistor} -- Valor nominal: 1 k$\Omega$
\item \textbf{Capacitor} -- Valor nominal: 0,22 $\mu F$
\item \textbf{Osciloscópio}
\item \textbf{Gerador de Ondas}
\item \textbf{Cabo BNC-banana}
\end{itemize}

\subsection{Componentes do circuito passa-banda e equipamentos:}

\begin{itemize}
	\item \textbf{Indutor} -- Valor nominal: 48,6 mH
	\item \textbf{Capacitor} -- Valor nominal: 0,44 $\mu F$
	\item \textbf{Osciloscópio}
	\item \textbf{Gerador de Ondas}
	\item \textbf{Cabo BNC-banana}
\end{itemize}

\begin{figure} [H]
\centering
\begin{circuitikz}[scale=1]
		\node (Xi) at (0.7,0.7) {$V_1$};
		\node (Xf) at (3.7,0.7) {$V_2$};
		\draw [semithick,->] (Xi) -- (0.1,0.1);
		\draw [semithick,->] (Xf) -- (3.1,0.1);
		\draw to [resistor, o-o, l_=$R$] ++(2,0)
			(2,0) to [short,o-o] ++(1,0)
			(2,0) to [capacitor, o-o, l=$C$] ++(0,-2)
			node[ground] {};
\end{circuitikz}
\caption{Circuito RC para o filtro passa-baixa}
\label{fig:rc}
\end{figure}

\begin{figure} [H]
\centering
\begin{circuitikz}[scale=1]
		\node (Xi) at (0.7,0.7) {$V_1$};
		\node (Xf) at (5.7,0.7) {$V_2$};
		\draw [semithick,->] (Xi) -- (0.1,0.1);
		\draw [semithick,->] (Xf) -- (5.1,0.1);
		\draw 
        	(0,0) to [resistor, o-o, l_=$R$] ++(2,0)
        	(2,0) to [capacitor, o-o, l_=$C$] ++(2,0)
			(4,0) to [short,o-o] ++(1,0)
			(4,0) to [inductor, o-o, l=$L$] ++(0,-2)
			node[ground] {};
\end{circuitikz}
\caption{Circuito RLC para o filtro passa-banda}
\label{fig:rlc}
\end{figure}

\subsection{Cálculo}

\subsubsection{Circuito RC (Passa-Baixa)}

\paragraph{Para o cálculo de $T_{dB}$} foi usado a equação:

$$
	H(\omega) = 
	\frac{V_2}{V_1} = 
    \frac{\frac{1}{j\omega C}}{R + \frac{1}{j\omega C}} =
    \frac{1}{1 + j\omega R C}
$$

$$
	T_{dB} = 
    20 log|H(\omega)| = 
    20 log(\frac{V_2}{V_1}) = 
    20 log(\frac{1}{\sqrt{1 - (\omega R C)^2}}) = 
    -10 log(1 - (\omega R C)^2)
$$

\paragraph{E para o cálculo da fase ($\Theta$)} foi usado a equação:

$$
	\Theta =
    -arg(H(\omega)) =
    -arctg(\omega R C) = 
    -arctg(j(1 - \frac{V_1}{V_2})) 
$$

\paragraph{Com suas respectivas propagações de erros}

$$
	\sigma_{T_{dB}} = \pm
    \sqrt{
    	(-\frac{V_2}{V_1^2})^2 \sigma_{V_1}^2 +
    	(\frac{1}{V_1})^2 \sigma_{V_2}^2
    }
$$

$$
	\sigma_\Theta = \pm
    \sqrt{
    	(\frac{V_2}{2 V_2 V_1 - V_1})^2 \sigma_{V_1}^2 +
    	(\frac{1}{V_2 - 2 V_1})^2 \sigma_{V_2}^2 
    }
$$

\subsubsection{Circuito RLC (Passa-Banda)}

\paragraph{Para o cálculo de $T_{dB}$} foi usado a equação:

$$
	H(\omega) = 
	\frac{V_2}{V_1} = 
    \frac{j \omega R C}{1 + \omega^2 L C - j \omega R C} =
    \frac{1}{1 - j\frac{1 - \omega^2 L C}{\omega R C}}
$$

$$
	T_{dB} = 
    20 log|H(\omega)| = 
    20 log(\frac{1}{\sqrt{1 - (\frac{1 - \omega^2 L C}{\omega R C})^2}})
$$

\paragraph{E para o cálculo da fase ($\Theta$)} foi usado a equação:

$$
	\Theta =
    -arctg(\frac{1 - \omega^2 L C}{\omega R C})
$$

\paragraph{Com suas respectivas propagações de erros}

$$
	\sigma_{T_{dB}} = \pm
    \sqrt{
    	(-\frac{V_2}{V_1^2})^2 \sigma_{V_1}^2 +
    	(\frac{1}{V_1})^2 \sigma_{V_2}^2
    }
$$

$$
	\sigma_\Theta = \pm
    \sqrt{
    	(\frac{V_2}{2 V_2 V_1 - V_1})^2 \sigma_{V_1}^2 +
    	(\frac{1}{V_2 - 2 V_1})^2 \sigma_{V_2}^2 
    }
$$

\section{Resultados} 

\begin{figure} [H] 
    \centering
    \pgfplotstableset{ 
        columns/f/.style={
            column name={$f [Hz]$},
        },
        columns/V1/.style={
            column name={$V_1 [V]$},
        },
        columns/e_V1/.style={
            column name={$\sigma_{V_1} [V]$},
        },
        columns/V2/.style={
            column name={$V_2 [V]$},
        },
        columns/e_V2/.style={
            column name={$\sigma_{V_2}[V]$},
        },
        columns/TdB/.style={
            column name={$T [dB]$},
        },
        columns/e_TdB/.style={
            column name={$\sigma_T [dB]$},
        },
        columns/fase/.style={
            column name={$\Theta$},
        },
        columns/e_fase/.style={
            column name={$\sigma_\Theta$},
        },
    }
    \pgfplotstableread{data/passa-baixa.dat}\loadedtable
    \pgfplotstabletypeset{\loadedtable}
    \caption{Tabela do circuito passa baixa}
    \label{fig:tbaixa}
\end{figure}

\begin{figure} [H] 
    \centering
    \pgfplotstableset{ 
        columns/f/.style={
            column name={$f [Hz]$},
        },
        columns/V1/.style={
            column name={$V_1 [V]$},
        },
        columns/e_V1/.style={
            column name={$\sigma_{V_1} [V]$},
        },
        columns/V2/.style={
            column name={$V_2 [V]$},
        },
        columns/e_V2/.style={
            column name={$\sigma_{V_2}[V]$},
        },
        columns/TdB/.style={
            column name={$T [dB]$},
        },
        columns/e_TdB/.style={
            column name={$\sigma_T [dB]$},
        },
        columns/fase/.style={
            column name={$\Theta$},
        },
        columns/e_fase/.style={
            column name={$\sigma_\Theta$},
        },
    }
    \pgfplotstableread{data/passa-banda.dat}\loadedtable
    \pgfplotstabletypeset{\loadedtable}
    \caption{Tabela do circuito passa banda}
    \label{fig:tbanda}
\end{figure}

\subsection{Diagramas de Bode}

Para cada filtro, por meio do script em Python disponibilizado em laboratório, foi possível obter os gráficos da Transmitância (medida em decibéis) e da resposta em fase do circuito (medida em graus) (Diagrama de Bode). Para ambos os gráficos, coletamos 15 pontos.

	\subsubsection{Passa Baixa}
		\begin{figure} [H] 
			\centering
			\begin{tikzpicture}
				\begin{axis}[
					width=12cm,
					xmode=log,
					ylabel={$transmissão [dB]$},
					xlabel style={below right},
					ylabel style={above left},
				]
				\addplot [color=red, mark=o, smooth, ultra thick]
					plot [error bars/.cd, y dir = both, y explicit]
					table[x=f, y=TdB, y error=e_TdB]{data/passa-baixa.dat};
				\end{axis}
			\end{tikzpicture}
			\begin{tikzpicture}
				\begin{axis}[
					width=12cm,
					xmode=log,
					xlabel={$frequência [Hz]$},
					ylabel={$fase [\circ]$},
					xlabel style={below right},
					ylabel style={above left},
				]
				\addplot [color=cyan, mark=o, smooth, ultra thick]
					plot [error bars/.cd, y dir = both, y explicit]
					table[x=f, y=fase, y error=e_fase]{data/passa-baixa.dat};
				\end{axis}
			\end{tikzpicture}
			\caption{Gráfico de Bode do circuito passa-baixa}
			\label{fig:graphD2}
		\end{figure}
	\subsubsection{Passa Banda}
		\begin{figure} [H] 
			\centering
			\begin{tikzpicture}
				\begin{axis}[
					width=12cm,
					xmode=log,
					ylabel={$transmissão [dB]$},
					xlabel style={below right},
					ylabel style={above left},
				]
				\addplot [color=red, mark=o, smooth, ultra thick]
					plot [error bars/.cd, y dir = both, y explicit]
					table[x=f, y=TdB, y error=e_TdB]{data/passa-banda.dat};
				\end{axis}
			\end{tikzpicture}
			\begin{tikzpicture}
				\begin{axis}[
					width=12cm,
                    xmode=log,
					xlabel={$frequência [Hz]$},
					ylabel={$fase [\circ]$},
					xlabel style={below right},
					ylabel style={above left},
				]
				\addplot [color=cyan, mark=o, smooth, ultra thick]
					plot [error bars/.cd, y dir = both, y explicit]
					table[x=f, y=fase, y error=e_fase]{data/passa-banda.dat};
				\end{axis}
			\end{tikzpicture}
			\caption{Gráfico de Bode do circuito passa-banda}
			\label{fig:graphD2}
		\end{figure}
        
\section{Discussão}

\subsection{Passa-baixa RC}

O filtro passa-baixa RC possui, como função de transferência (H(W)), a seguinte expressão: $H(w) = \frac{1}{ 1 - jwRC}$ . Assim, a transmitância, em dB, é dada por $T = 20\log (\frac{1}{\sqrt{(1 + (wRC)^2}})$. A fase, por sua vez, é o dada por $\theta = -arctg(wRC)$ e a frequência de corte, dada por $f_{c} = \frac{1}{RC}$. A partir dessas expressões, é possível significar o que se passa no diagrama de Bode do filtro passa baixa: Para frequências maiores que a frequência de corte (w >> wc), ou seja, sinais de alta frequência o capacitor apresenta baixa reatância, XC << R e seu comportamento é o de um curto-circuito. Logo, grande parte da tensão de entrada
estará sobre o resistor e a tensão sobre o capacitor de saída será quase nula. Assim, o circuitocimpede a passagem de sinais de alta frequência - o que é perceptível pela atenuação crescente após a frequencia de corte ($f_{c} = \frac{1}{RC} = 0,22.10^{-3} Hz$) na curva de transmitância. Por sua vez, para frequências menores que a frequência de corte, ($w << w_{c}$), o capacitor apresenta alta reatância, XC >> R e seu comportamento é o de um circuito aberto. Desta forma, grande parte da tensão de entrada estará sobre o capacitor de saída. Logo, o circuito deixa passar sinais
de baixa frequência.  

\subsection{Passa-banda RLC}

O filtro passa-banda RLC possui, como função de transferência (H(w)), a seguinte expressão $H(w) =\frac{1}{1 -  (j\frac{1 - w^2LC}{wRC})}$. A transmitância, em dB, é dada por $T = 20 \log (\frac{1}{\sqrt{1 + \frac{1 - w^2LC}{wRC}^2}})$. A fase é dada por $\theta = arctg \frac{1 - w^2LC}{wRC}^2 $. No circuito passa baixa, temos duas frequências de corte e uma frequência central, dadas, respectivamente, por  $w_{c1} = \frac{-RC +- \sqrt{RC^2 + 4LC}}{2LC}  w_{c2} = \frac{+RC +- \sqrt{RC^2 + 4LC}}{2LC}  w_{central} = \sqrt{\frac{1}{LC}} $  . 
Traduzindo esses conceitos para o diagrama de bode, temos que para frequências menores que a frequência de corte 1 ($w << w_{c1}$): o indutor do circuito da apresenta baixa reatância indutiva e tende a comportar-se como um curto-circuito, porém, o capacitor apresenta alta reatância capacitiva e tende a comportar-se como um circuito aberto. Desta forma, grande parte da tensão de entrada estará sobre o capacitor e a tensão sobre o resistor de saída será muito baixa, ou seja, o sinal será atenuado. Assim, o circuito impede a passagem de sinais de baixa frequência.

Para frequências maiores que a frequência de corte 2 ($w >> w_{c2}$):  o capacitor apresenta baixa reatância capacitiva e comporta-se como um curto-circuito, porém, o indutor apresenta alta reatância indutiva e comporta-se como um circuito aberto. Desta forma, grande parte de tensão de entrada estará
sobre o indutor e a tensão sobre o resistor de saída será muito baixa, ou seja, o sinal será atenuado. Portanto, o circuito impede a passagem de sinais de alta frequência.  

Para sinais de frequências intermediárias, ou seja, sinais cujas frequências estiverem numa faixa
próxima à frequência central do filtro, o indutor e o capacitor juntos apresentarão
baixa reatância e tenderão a comportarem-se como um curto circuito. Desta forma, grande parte da tensão de entrada estará sobre o resistor de
saída. Deste modo o circuito deixa passar sinais dentro de uma determinada faixa
de frequência. 

\section{Referências}

Análise de Circuitos - Teoria e Prática - Vol. 2 -
Miller,Wilhelm C. / Robbins,Allan H.
        
\end{document}